
% --------------- 12 POINT FONT -------------------------------
\documentclass[12pt]{article}
% --------------- 10 POINT FONT FOR CAPTIONS ------------------
\usepackage[font=footnotesize]{caption}
% --------------- NY TIMES FONT -------------------------------
\usepackage{times}
% --------------- 1 INCH MARGINS ------------------------------
\usepackage[margin=1in]{geometry}
% --------------- LINE SPACING --------------------------------
\usepackage{setspace}
\singlespacing
%\doublespacing
% --------------- SMALL SECTION TITLES ------------------------
\usepackage[tiny,compact]{titlesec}
% --------------- PACKAGES ------------------------------------
\usepackage{bookmark}
\usepackage{algorithm}
\usepackage{algpseudocode}
\usepackage{amsfonts}
\usepackage{amsmath}
\usepackage{amssymb}
\usepackage{amsthm}
\usepackage{bm}
\usepackage{color}
\usepackage{comment}
\usepackage{float}
\usepackage{graphicx}
%\usepackage[hidelinks]{hyperref}
\usepackage{makecell}
\usepackage[caption=false,font=footnotesize,subrefformat=parens,labelformat=parens]{subfig}
\usepackage{wrapfig}
\usepackage{url}
\usepackage[table]{xcolor}
\begin{document}
% --------------- TITLE AND NAME ------------------------------
\begin{center}
\textbf{Summary}\\
\end{center}

\noindent
Bardia Mojra\\
\today\\
Seminar on Continual Learning\\
Robotic Vision Lab\\
% --------------- CONTENT -------------------------------------
\begin{center}
Unified Probabilistic Deep Continual Learning\\
through Generative Replay and Open Set Recognition\\
\end{center}

This would be the first line. Please read \cite{mur2015orb}.\\

$$
P(z | D, \alpha) = \frac{P(z,D\|\alpha)}{\int_{z} P(z,D \| \alpha)}
$$


%Sets the bibliography style to UNSRT and import the
%\newpage
\bibliography{ref}
\bibliographystyle{ieeetr}

\end{document}
